%
% Daftar Pustaka 
% 

% 
% Tambahkan pustaka yang digunakan setelah perintah berikut. 
% 
%\begin{thebibliography}

%\bibitem{per_journal}
%{J. K. Author, “Name of paper,” \textit{Abbrev. Title of Periodical}, vol. x, no. x, pp. xxx-xxx, Abbrev. Month, year.}
%
%\bibitem{per_journal_1}
%{S. Azodolmolky et al., Experimental demonstration of an impairment aware network planning and operation tool for transparent/translucent optical networks ,” \textit{J. Lightw. Technol.}, vol. 29, no. 4, pp. 439–448, Sep. 2011.}
%
%\bibitem{per_journal_2}
%{J. Zhang and N. Tansu, “Optical gain and laser characteristics of InGaN quantum wells on ternary InGaN substrates,” \textit{IEEE Photon. J.}, vol. 5, no. 2, Apr. 2013, Art. ID 2600111.}
%
%\bibitem{per_journal_3}
%{W. Rafferty, “Ground antennas in NASA’s deep space telecommunications,” \textit{Proc. IEEE}, vol. 82, no. 5, pp. 636-640, May 1994.}
%
%\bibitem{book}
%{J. K. Author, “Title of chapter in the book,” in \textit{Title of His Published Book}, xth ed. City of Publisher, (only U.S. State), Country: Abbrev. of Publisher, year, ch. x, sec. x, pp. xxx–xxx.}
%
%\bibitem{book_1}
%{R. Ramaswami, K. N. Sivarajan, and G. H. Sasaki, \textit{Optical Network: A Practical Prespective}, 3rd ed. Burlington, MA, USA: Elsevier, 2010.}
%
%\bibitem{report}
%{J. K. Author, “Title of report,” Abbrev. Name of Co., City of Co., Abbrev. State, Country, Rep. xxx, year.}
%
%\bibitem{report_1}
%{M. Chui et al, "The Social Economy: Unlocking Value and Productivity Through Social Technologies", McKinsey Global Institute, 2012.}
%
%\bibitem{conference}
%{J. K. Author, “Title of paper,” in \textit{Abbreviated Name of Conf}., (location of conference is optional), year, pp. xxx-xxx.}
%
%\bibitem{conference_1}
%{B. Lantz, B. Heller dan N. McKeown, "A Network in a Laptop: Rapid Prototyping for Software-defined Networks"., dalam \textit{Proceedings of the 9th ACM SIGCOMM Workshop on Hot Topics in Networks}, New York, 2010.}
%
%\bibitem{conference_2}
%{A. Sefano, D. Emma, A. Pescape dan G. Ventre, "A Practical Demonstration of Network Traffic Generation"., dalam \textit{Proceedings of the 8th IASTED International Conference}, Hawaii, 2004, pp. 138-143.}
%
%\bibitem{paten}
%{J. K. Author, “Title of patent,” U.S. Patent x xxx xxx, Abbrev. Month, day, year.}
%
%\bibitem{paten_1}
%{S. P. Voinigescu et al., "Direct m-ary quadrature amplitude modulation (QAM) operating in saturated power mode,” U.S. Patent Appl. 20110013726A1, Jan. 20, 2011.}
%
%\bibitem{theses}
%{J. K. Author, “Title of thesis,” M.S. thesis, Abbrev. Dept., Abbrev. Univ., City of Univ., Abbrev. State, year.}
%
%\bibitem{dissertation}
%{J. K. Author, “Title of dissertation,” Ph.D. dissertation, Abbrev. Dept., Abbrev. Univ., City of Univ., Abbrev. State, year.}
%
%\bibitem{theses_1}
%{N. Kawasaki, “Parametric study of thermal and chemical nonequilibrium nozzle flow,” M.S. thesis, Dept. Electron.
%Eng., Osaka Univ., Osaka, Japan, 1993.}
%
%\bibitem{theses_2}
%{B. Joaquim, "Redundancy and Load Balancing at IP layer in Access and Aggregation Networks", Master Thesis, Aalto University. 2011}
%
%\bibitem{thesis_2}
%{S. Ejaz, "Analysis of the trade-off between performance and energy consumption of existing load balancing algorithms", Grober Beleg, Technische Universitat Dresden, 2011.}
%
%\bibitem{standard}
%{\textit{Title of Standard}, Standard number, date.}
%
%\bibitem{standard_1}
%{\textit{Software-Defined Networking: The New Norm for Networks}, Open Network Foundation, 2012.}
%
%\bibitem{standard_2}
%{\textit{Virtual Router Redundancy Protocol (VRRP) Version 3 for IPv4 and IPv6}. IETF RFC 5798. 2010.}
%
%\bibitem{manual}
%T. Antcom, CA, USA. Antenna Products. (2011) [Online]. Tersedia di http://www.antcom.com/documents/catalogs /L1L2GPSAntennas.pdf, Diakses pada 12 Februari 2014.
%
%\bibitem{software_source_code}
%Drox. (2015) [Online]. Tersedia di \url{http://github.com/haidlir/drox}, Diakses pada 26 Juli 2015.
%
%%\bibitem{Artikel}
%%{T. Oetiker. The Not So Short Introduction to \latex2ε. (2014) [Online]. Tersedia di \url{https://tobi.oetiker.ch/lshort/lshort.pdf}. 
%%Diakses pada 8 Mei 2015.}
%
%\bibitem{ta_panduan}
%{Format Penulisan Buku Tugas Akhir S1. (2011) [Online]. Tersedia di \url{http://www.ee.itb.ac.id/format_penulisan_buku_tugas_akhir_s1}. Diakses pada 26 Juli 2015.}
%
%\bibitem{ieee_style}
%{IEEE Editorial Style Manual. (2014) [Online]. Tersedia di \url{https://www.ieee.org/documents/style_manual.pdf}. Diakses pada 25 Juli 2015.}



\bibliography{refer}  
\bibliographystyle{IEEEtran}
%\end{thebibliography}