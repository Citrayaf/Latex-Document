
\chapter*{abstrak}
\vspace*{0.7cm}



Tugas Akhir ini melakukan studi pada \textit{Low Density Parity Check} (LDPC) \textit{codes} \textit{Digital Video Broadcasting - Second Generation Terrestrial} DVB-T2 untuk mendapatkan struktur dan kinerja dengan \textit{channel model} Indonesia. Langkah pertama pada Tugas Akhir ini adalah melakukan pengujian \textit{code rate} LDPC \textit{codes} dari standar DVB-T2 pada \textit{channel model} Indonesia. Pengujian dilakukan dengan simulasi komputer menggunakan struktur LDPC \textit{codes} dari standar DVB-T2 sehingga kinerja setiap \textit{code rate} yang berpeluang menjadi standar TV digital Indonesia bisa diketahui. Tugas Akhir ini juga mengusulkan modifikasi LDPC \textit{codes} DVB-T2 untuk menjadi standar LDPC \textit{codes} pada DVB-T2 \text{Indonesia} dengan menggunakan metode \textit{Extrinsic Information Transfer} (EXIT) \textit{analysis}.
%Tugas Akhir ini menguji dan menganalasis perfomansi berbagai \textit{rate} LDPC \textit{Codes} yang telah ditetapkan ETSI sebagai standar DVB-T2.   \textit{Rate} akan dianalisis performansinya dengan parameter \textit{Bit Error Rate} (BER) dan \textit{Block Error Rate} (BLER) yang terbaik. 


	Untuk mengurangi kompleksitas proses komputasi pada \textit{encoder} dan \textit{decoder}, Tugas Akhir ini mengusulkan teknik \textit{downscaling} dengan dan tanpa algoritma \textit{Progresive Edge-Growth} (PEG) untuk LDPC \textit{codes} DVB-T2. Teknik \textit{downscaling} memungkinkan untuk memperpendek panjang LDPC \textit{codes}, sehingga LDPC \textit{codes} DVB-T2 dengan panjang blok 16200 dapat diperkecil menjadi hanya 270. \textit{Downscaled} LDPC \textit{codes} juga diharapkan dapat digunakan untuk perangkat lain dengan daya dan kompleksitas yang rendah.


Tugas Akhir ini menghasilkan beberapa poin berikut ini: (i) struktur LDPC \textit{codes} dengan panjang blok 16200 dan 270 untuk setiap \textit{code rate} yang sesuai \text{dengan} standar DVB-T2, (ii) teknik perancangan LDPC \textit{codes} menggunakan algoritma PEG tanpa adanya \textit{girth} 4 sehingga menjamin \textit{error} yang rendah karena tidak terjadi \textit{cycle} yang melibatkan 4 \textit{nodes}, (iii) teknik untuk menghitung \textit{girth} pada LDPC \textit{codes} yang bermanfaat untuk desain LDPC berbagai ukuran, (iv) kinerja yang baik dari LDPC \textit{codes} DVB-T2 pada kanal \textit{Additive White Gaussian Noise} (AWGN) dan \textit{frequency-selective fading} dengan menggunakan \textit{channel model} DVB-T2 Indonesia. Hasil Tugas Akhir ini diharapkan juga dapat membantu optimalisasi DVB-T2 Indonesia, serta membantu pengembangan LDPC \textit{codes} yang berukuran kecil untuk perangkat berdaya dan kompleksitas rendah, seperti perangkat \textit{Internet of Things} (IoT) dan drone.
%\noindent Abstrak ini\\
%
%{\color{red}nama saya} \\
%\underline{Citra Yasin Akbar Fadhlika}
%\vspace{2cm} %vertikal space
%%\hspace{} untuk horizontal space
%
%Saya tinggal \textit{\textbf{diiiiii}}
%\vspace*{0.2cm}

\noindent Kata Kunci: \textit{\textbf{Error correction coding}}, \textit{\textbf{DVB-T2}}, \textbf{\textit{LDPC} \textit{codes}}, \textbf{\textit{code rate}}

