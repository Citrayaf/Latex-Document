
\chapter*{ABSTRACT}
\vspace*{0.7cm}

This thesis studies Low Density Parity Check (LDPC) codes of Digital Video Broadcasting Second Generation (DVB-T2) to obtain good structure and performances for Indonesia DVB-T2 channel model. The first step of this thesis \text{simulates} performances of LDPC codes from DVB-T2 standard in Indonesia DVB-T2 channel model using computer-based simulation. All the code rates of DVB-T2 LDPC codes are evaluated, of which has a chance to become parameter for digital television standard in Indonesia can be known. This thesis also propose a modified DVB-T2 LDPC codes for standard of Indonesia.

 To reduce the computational complexity of encoder and decoder, this thesis uses a downscaling technique and proposes downscaling technique using Progresive Edge-Growth (PEG) algorithm for LDPC codes of DVB-T2 with a reduced block length of 16200 bits to 270 bits. The results of downscaled LDPC codes are also expected to be used for devices consuming low power and low complexity such as device for Internet of Things (IoT) and drones. 

This thesis obtained following results: (i) the structure of DVB-T2 LDPC codes with block length 16200 and 270 for each code rate, (ii) a technique for designing LDPC codes without girth 4 for LDPC codes construction, (iii) proposed technique to calculate girth from LDPC codes, and (iv) the simulation results showing \text{acceptable} Bit Error Rate (BER) performances of DVB-T2 LDPC codes under \text{Additive} White Gaussian Noise (AWGN) and frequency selective fading channel using Indonesia DVB-T2 channel model. The results of this thesis are expected to fasten the DVB-T2 implementation in Indonesia and assist in the development of small-sized LDPC codes for devices with low power and complexity.

%
% Tugas Akhir ini melakukan studi LDPC \textit{codes} DVB-T2 untuk mendapatkan struktur dan nilai \textit{code rate} yang sesuai dengan \textit{channel model} Indonesia. Langkah pertama adalah melakukan pengujian \textit{code rate} LDPC \textit{codes} dari standar DVB-T2 pada \textit{channel model} Indonesia. Pengujian dilakukan dengan simulasi komputer menggunakan struktur LDPC \textit{codes} dari standar DVB-T2 sehingga \textit{code rate} yang terbaik akan diusulkan untuk menjadi standar TV digital Indonesia. Apabila hasil evaluasi ini menunjukkan bahwa semua \textit{code rate} tidak sesuai dengan \textit{channel model} \text{Indonesia}, maka langkah kedua Tugas Akhir ini mengusulkan modifikasi LDPC \textit{codes} DVB-T2 untuk menjadi standar LDPC \textit{codes} pada DVB-T2 \text{Indonesia} dengan menggunakan metode \textit{Extrinsic Information Transfer} (EXIT) \textit{chart}.
%%Tugas Akhir ini menguji dan menganalasis perfomansi berbagai \textit{rate} LDPC \textit{Codes} yang telah ditetapkan ETSI sebagai standar DVB-T2.   \textit{Rate} akan dianalisis performansinya dengan parameter \textit{Bit Error Rate} (BER) dan \textit{Block Error Rate} (BLER) yang terbaik. 
%
%Hasil yang diharapkan dari Tugas Akhir ini adalah: (i) struktur LDPC \textit{codes} dan nilai \textit{code rate} yang sesuai dengan \textit{channel model} Indonesia sehingga kinerja FEC DVB-T2 optimal dan (ii) kinerja LDPC \textit{codes} DVB-T2 pada \textit{Additive White Gaussian Noise} (AWGN) dan \textit{frequency-selective fading} \textit{channel} menghasilkan nilai \textit{Bit Error Rate} (BER) kurang dari $10^{-3}$. Hasil Tugas Akhir diharapkan juga dapat membantu proses pembuatan standar DVB-T2 Indonesia sehingga dapat mempercepat migrasi DVB-T ke DVB-T2 di Indonesia.
%\noindent Abstrak ini\\
%
%{\color{red}nama saya} \\
%\underline{Citra Yasin Akbar Fadhlika}
%\vspace{2cm} %vertikal space
%%\hspace{} untuk horizontal space
%
%Saya tinggal \textit{\textbf{diiiiii}}
%\vspace*{0.2cm}

\noindent Keywords: \textit{\textbf{Error correction coding}}, \textit{\textbf{DVB-T2}}, \textbf{\textit{LDPC} \textit{codes}}, \textbf{\textit{code rate}}\\ 